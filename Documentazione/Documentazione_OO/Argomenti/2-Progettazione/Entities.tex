\section{Analisi delle entità}
Le entità che possono essere individuate nel problema sono:
\begin{enumerate}
\item \textbf{Conferenza} : per le conferenze delle quali si vuole poter gestire le informazioni. Di ogni conferenza si conservano il \textit{nome}, l'\textit{inizio} e la \textit{fine} e una \textit{descrizione}.
\item \textbf{Ente}: per gli enti che organizzano le conferenze scientifiche. Di ogni ente si conserva il \textit{nome} e la \textit{sigla}. 
\item \textbf{Sponsor} : per gli sponsor che coprono le spese della conferenza. Di ogni sponsor si conserva il \textit{nome}.
\item \textbf{Comitato} : per i gruppi di organizzatori che si occupano della gestione della conferenza. Si distinguono in comitati \textit{scientifici} e \textit{locali}. 
\item \textbf{Organizzatore} : per i membri dei comitati. Di ogni organizzatore si riportano \textit{titolo, nome, cognome, email} ed \textit{istituzione di afferenza}.
\item \textbf{Sede}: per descrivere il luogo dove si tengono le varie conferenze. Di ogni sede si conservano il \textit{nome}, l'\textit{indirizzo} e la \textit{città}.
\item \textbf{Sala}: per tenere traccia dell'ubicazione delle varie sessioni. Di ogni sala si conserva il \textit{nome della sala} e la sua \textit{capacità}.
\item \textbf{Sessione}: per rappresentare le sessioni di una conferenza. Per ogni sessione si riporta il \textit{titolo}, un \textit{coordinatore}, data e orario d'\textit{inizio} e di \textit{fine}.
\item \textbf{Programma} : per il programma di ciascuna sessione. Ogni programma specifica la presenza di un \textit{keynote speaker}, ovvero un partecipante di rilievo.
\item \textbf{Intervento} : per i vari interventi di una sessione. Per ogni intervento si conserva un \textit{abstract} e l'\textit{orario} dello stesso.
\item \textbf{Speaker}:per descrivere chi effettua un intervento. 
\item \textbf{Partecipante}: per i partecipanti delle varie sessioni. Ogni partecipante ha gli stessi attributi degli organizzatori
\item \textbf{Intervallo}: per descrivere i vari intervalli presenti all'interno di una sessione. Questi possono essere di tue tipologie: \item \textit{coffee break} oppure dei \textit{pranzi}. Per ogni intervallo si riporta l'\textit{orario}. 
\item \textbf{Evento sociale}: per i vari eventi sociali previsti all'interno di una sessione. Questi possono essere di varia natura. Come per gli intervalli se ne riporta l'\textit{orario}.
\item \textbf{Utente}: per i vari utenti che creano le conferenze all'interno di un applicativo. 
\end{enumerate}