\newpage
\subsection{Il package \texttt{Controller}}
Nel contesto dello sviluppo di applicazioni JavaFX come Symposium, i controller svolgono un ruolo centrale nell'architettura dell'applicazione. I controller fungono da punto di collegamento tra l'interfaccia utente (View) e dati sottostante (Model), ma anche con le classi di utilities e le classi DAO. Questa relazione ben strutturata tra controller e altre componenti è essenziale per la realizzazione di applicazioni robuste ed estensibili.
\bigskip

I controller, infatti, gestiscono la business logic dell'intera applicazione essendo responsabili di:
\begin{enumerate}
	\item \textbf{Gestire l'interfaccia utente}: i controller gestiscono gli elementi dell'interfaccia utente, come finestre, pulsanti, campi di testo e altro ancora. Rispondono agli eventi generati dagli utenti, come clic del mouse o pressione dei tasti, e reagiscono modificando la vista o invocando operazioni sul Model.
	\item \textbf{Comunicare con il modello:} i controller accedono alle classi di modello per ottenere o modificare dati. Ad esempio, quando un utente inserisce dati in un campo di input, il controller può recuperare questi dati dal Model e inviarli per l'elaborazione.
	\item \textbf{Gestire la logica di business:} i controller implementano la logica di business dell'applicazione. Questa logica può includere la validazione dei dati, il calcolo di risultati o qualsiasi altra elaborazione necessaria.
\end{enumerate}

\subsubsection{Controller.Edit}
I controller presenti nel package \texttt{Controller.Edit} si occupano della gestione e modifica di tutto ciò che riguarda una conferenza. Dalla gestione della durata, alla gestione delle singole sessioni previste.
\begin{table}[h!]
	\begin{tabular}{|l|l|}
		\hline 
		\textbf{Class name} & \texttt{AddSessione\_Controller}
		\\ \hline
		\textbf{Superclass} & \texttt{ViewAlert\_Controller}
		\\ \hline
		% Sostituire ad 1 il numero di classi figlie
		\multirow{1}{*}{\textbf{Subclasses}} & Nessuna
		\\ \hline
		\textbf{Responsability} & Finestra per l'inserimento di una sessione
		\\ \hline
		\multirow{1}{*}{\textbf{Collaborations}} & \texttt{Sessione} \\
		& \texttt{Sale} \\
		& \texttt{Conferenza} \\ \hline
	\end{tabular}
\end{table}
\begin{table}[h!]
	\begin{tabular}{|l|l|}
		\hline 
		\textbf{Class name} & \texttt{ChooseKeynote\_Controller}
		\\ \hline
		\textbf{Superclass} & \texttt{ViewAlert\_Controller}
		\\ \hline
		% Sostituire ad 1 il numero di classi figlie
		\multirow{1}{*}{\textbf{Subclasses}} & Nessuna
		\\ \hline
		\textbf{Responsability} & Finestra di selezione del keynote
		\\ \hline
		\multirow{1}{*}{\textbf{Collaborations}} & \texttt{Programma}
		\\ \hline
	\end{tabular}
\end{table}
\begin{table}[h!]
	\begin{tabular}{|l|l|}
		\hline 
		\textbf{Class name} & \texttt{EditKeynote\_Controller}
		\\ \hline
		\textbf{Superclass} & \texttt{ViewAlert\_Controller}
		\\ \hline
		% Sostituire ad 1 il numero di classi figlie
		\multirow{1}{*}{\textbf{Subclasses}} & Nessuna
		\\ \hline
		\textbf{Responsability} & Finestra per la modifica del keynote di una sessione
		\\ \hline
		\multirow{2}{*}{\textbf{Collaborations}} & \texttt{Sessione} \\
		& Programma \\ \hline
	\end{tabular}
\end{table}
\begin{table}[h!]
	\begin{tabular}{|l|l|}
		\hline 
		\textbf{Class name} & \texttt{ModificaConferenza\_Controller}
		\\ \hline
		\textbf{Superclass} & \texttt{ViewAlert\_Controller}
		\\ \hline
		% Sostituire ad 1 il numero di classi figlie
		\multirow{1}{*}{\textbf{Subclasses}} & Nessuna
		\\ \hline
		\textbf{Responsability} & Visualizzazione dettagli conferenza
		\\ \hline
		\multirow{1}{*}{\textbf{Collaborations}} &\texttt{ Conferenza}
		\\ \hline
	\end{tabular}
\end{table}

\begin{table}[h!]
	\begin{tabular}{|l|l|}
		\hline 
		\textbf{Class name} &\texttt{ ModificaConferenze\_Controller}
		\\ \hline
		\textbf{Superclass} & \texttt{ViewAlert\_Controller}
		\\ \hline
		% Sostituire ad 1 il numero di classi figlie
		\multirow{1}{*}{\textbf{Subclasses}} & Nessuna
		\\ \hline
		\textbf{Responsability} & Visualizzazioni conferenze dell'utente
		\\ \hline
		\multirow{1}{*}{\textbf{Collaborations}} & \texttt{ConferenzeUtente}
		\\ \hline
	\end{tabular}
\end{table}
\begin{table}[h!]
	\begin{tabular}{|l|l|}
		\hline 
		\textbf{Class name} & \texttt{ModificaDettagliConferenza\_Controller}
		\\ \hline
		\textbf{Superclass} & \texttt{ViewAlert\_Controller}
		\\ \hline
		% Sostituire ad 1 il numero di classi figlie
		\multirow{1}{*}{\textbf{Subclasses}} & Nessuna
		\\ \hline
		\textbf{Responsability} & Gestione informazioni conferenza
		\\ \hline
		\multirow{2}{*}{\textbf{Collaborations}} & \texttt{Conferenza} \\
		& \texttt{Sedi }
		\\ \hline
	\end{tabular}
\end{table}

\begin{table}[h!]
	\begin{tabular}{|l|l|}
		\hline 
		\textbf{Class name} & \texttt{ModificaDettagliSessione\_Controller}
		\\ \hline
		\textbf{Superclass} & \texttt{ViewAlert\_Controller}
		\\ \hline
		% Sostituire ad 1 il numero di classi figlie
		\multirow{1}{*}{\textbf{Subclasses}} & Nessuna
		\\ \hline
		\textbf{Responsability} & Modifica della durata, locazione e coordinatore di una sessione
		\\ \hline
		\multirow{3}{*}{\textbf{Collaborations}} & \texttt{Sessione} \\
		& \texttt{Sala} \\
		&\texttt{ Conferenza}
		\\ \hline
	\end{tabular}
\end{table}

\begin{table}[h!]
	\begin{tabular}{|l|l|}
		\hline 
		\textbf{Class name} & \texttt{ModificaEntiOrganizzatori\_Controller}
		\\ \hline
		\textbf{Superclass} & \texttt{ViewAlert\_Controller}
		\\ \hline
		% Sostituire ad 1 il numero di classi figlie
		\multirow{1}{*}{\textbf{Subclasses}} & Nessuna
		\\ \hline
		\textbf{Responsability} & Gestione delle istituzioni organizzatrici
		\\ \hline
		\multirow{2}{*}{\textbf{Collaborations}} &\texttt{ Enti }\\
		& \texttt{Conferenza	}	\\ \hline
	\end{tabular}
\end{table}
\begin{table}[h!]
	\begin{tabular}{|l|l|}
		\hline 
		\textbf{Class name} & \texttt{ModificaProgrammaSessione\_Controller}
		\\ \hline
		\textbf{Superclass} & \texttt{ViewAlert\_Controller}
		\\ \hline
		% Sostituire ad 1 il numero di classi figlie
		\multirow{1}{*}{\textbf{Subclasses}} & Nessuna
		\\ \hline
		\textbf{Responsability} & Gestione dei punti presenti nel programma
		\\ \hline
		\multirow{2}{*}{\textbf{Collaborations}} & \texttt{Programma} \\
		& \texttt{Sessione} 		\\ \hline
	\end{tabular}
\end{table}
\begin{table}[h!]
	\begin{tabular}{|l|l|}
		\hline 
		\textbf{Class name} & \texttt{ModificaSessione\_Controller}
		\\ \hline
		\textbf{Superclass} & \texttt{ViewAlert\_Controller}
		\\ \hline
		% Sostituire ad 1 il numero di classi figlie
		\multirow{1}{*}{\textbf{Subclasses}} & Nessuna
		\\ \hline
		\textbf{Responsability} & Gestione dei dettagli e del programma della sessione
		\\ \hline
		\multirow{2}{*}{\textbf{Collaborations}} & \texttt{Conferenza} \\
		& \texttt{Programma} \\ \hline
	\end{tabular}
\end{table}
\begin{table}[h!]
	\begin{tabular}{|l|l|}
		\hline 
		\textbf{Class name} &\texttt{ ModificaSessioni\_Controller}
		\\ \hline
		\textbf{Superclass} & \texttt{ViewAlert\_Controller}
		\\ \hline
		% Sostituire ad 1 il numero di classi figlie
		\multirow{1}{*}{\textbf{Subclasses}} & Nessuna
		\\ \hline
		\textbf{Responsability} & Finestra per aggiunta, rimozione e modifica delle sessioni presenti in una conferenza
		\\ \hline
		\multirow{1}{*}{\textbf{Collaborations}} & \texttt{Conferenza}
		\\ \hline
	\end{tabular}
\end{table}
\begin{table}[h!]
	\begin{tabular}{|l|l|}
		\hline 
		\textbf{Class name} & \texttt{ModificaSponsorizzazioni\_Controller}
		\\ \hline
		\textbf{Superclass} & \texttt{ViewAlert\_Controller}
		\\ \hline
		% Sostituire ad 1 il numero di classi figlie
		\multirow{1}{*}{\textbf{Subclasses}} & Nessuna
		\\ \hline
		\textbf{Responsability} & Aggiunta e rimozione di una sponsorizzazione
		\\ \hline
		\multirow{2}{*}{\textbf{Collaborations}} & \texttt{Conferenza} \\
		& \texttt{Sponsors	}	\\ \hline
	\end{tabular}
\end{table}
\clearpage
\pagebreak

\subsubsection{Controller.Create}
I controller presenti nel package \texttt{Controller.Create} si occupano della creazione di una conferenza. In particolare gestiscono il caricamento delle varie finestre FXML, l'elaborazione dei dati forniti dall'utente e le chiamate ai metodi di salvataggio dei dati.
\begin{table}[h!]
    \begin{tabular}{|l|l|}
        \hline 
        \textbf{Class name} & \texttt{AddConference\_Controller}
        \\ \hline
        \textbf{Superclass} & \texttt{ViewAlert\_Controller}
        \\ \hline
        % Sostituire ad 1 il numero di classi figlie
        \multirow{1}{*}{\textbf{Subclasses}} & Nessuna
        \\ \hline
        \textbf{Responsability} & Aggiunta delle prime informazioni su una conferenza
        \\ \hline
        \multirow{5}{*}{\textbf{Collaborations}} & \texttt{Conferenza} \\
		& \texttt{Sedi} \\
		& \texttt{ConferenzaDao} \\
		& \texttt{AddEnti\_Controller} \\
		& \texttt{Landing\_Controller}
        \\ \hline
    \end{tabular}
\end{table}

\begin{table}[h!]
	\begin{tabular}{|l|l|}
		\hline 
		\textbf{Class name} & \texttt{AddEnti\_Controller}
		\\ \hline
		\textbf{Superclass} & \texttt{ViewAlert\_Controller}
		\\ \hline
		% Sostituire ad 1 il numero di classi figlie
		\multirow{1}{*}{\textbf{Subclasses}} & Nessuna
		\\ \hline
		\textbf{Responsability} & Aggiunta e rimozione degli enti organizzatori della conferenza
		\\ \hline
		\multirow{5}{*}{\textbf{Collaborations}} & \texttt{Ente} \\
		& \texttt{Conferenza}\\
		& \texttt{Utente}\\
		& \texttt{ModificaConferenza\_Controller}\\
		& \texttt{AddComitati\_Controller}
		\\ \hline
	\end{tabular}
\end{table}

\begin{table}[h!]
	\begin{tabular}{|l|l|}
		\hline 
		\textbf{Class name} & \texttt{AddComitati\_Controller}
		\\ \hline
		\textbf{Superclass} & \texttt{ViewAlert\_Controller}
		\\ \hline
		% Sostituire ad 1 il numero di classi figlie
		\multirow{1}{*}{\textbf{Subclasses}} & Nessuna
		\\ \hline
		\textbf{Responsability} & Aggiunta e rimozione dei membri del comitato Scientifico e Locale
		\\ \hline
		\multirow{6}{*}{\textbf{Collaborations}} & \texttt{Conferenza} \\
		& \texttt{Ente} \\
		& \texttt{Organizzatore} \\
		& \texttt{Utente} \\
		& \texttt{AddEnti\_Controller} \\
		& \texttt{AddSponsor\_Controller}
		\\ \hline
	\end{tabular}
\end{table}

\begin{table}[h!]
	\begin{tabular}{|l|l|}
		\hline 
		\textbf{Class name} & \texttt{AddSponsor\_Controller}
		\\ \hline
		\textbf{Superclass} & \texttt{ViewAlert\_Controller}
		\\ \hline
		% Sostituire ad 1 il numero di classi figlie
		\multirow{1}{*}{\textbf{Subclasses}} & Nessuna
		\\ \hline
		\textbf{Responsability} & Aggiunta e rimozione delle aziende sponsor della conferenza
		\\ \hline
		\multirow{5}{*}{\textbf{Collaborations}} & \texttt{Conferenza }\\
		& \texttt{Sponsors}\\
		& \texttt{Sponsorizzazione}\\
		& \texttt{AddComitati\_Controller}\\
		& \texttt{VisualizzaSessioniConferenza\_Controller}
		\\ \hline
	\end{tabular}
\end{table}

\begin{table}[h!]
	\begin{tabular}{|l|l|}
		\hline 
		\textbf{Class name} & \texttt{VisualizzaSessioniConferenza\_Controller}
		\\ \hline
		\textbf{Superclass} & \texttt{ViewAlert\_Controller}
		\\ \hline
		% Sostituire ad 1 il numero di classi figlie
		\multirow{1}{*}{\textbf{Subclasses}} & Nessuna
		\\ \hline
		\textbf{Responsability} & Gestisce le sessioni della conferenza e fornisce diverse funzionalità di navigazione
		\\ \hline
		\multirow{8}{*}{\textbf{Collaborations}} & \texttt{Conferenza} \\
		& \texttt{Sessione}\\
		& \texttt{Utente}\\
		& \texttt{Organizzatore}\\
		& \texttt{AddSponsor\_Controller}\\
		& \texttt{InserisciSessione\_Controller}\\
		& \texttt{ViewProgramma\_Controller}\\
		& \texttt{EditConferenza\_Controller}
		\\ \hline
	\end{tabular}
\end{table}

\begin{table}[h!]
	\begin{tabular}{|l|l|}
		\hline 
		\textbf{Class name} & \texttt{InserisciSessione\_Controller}
		\\ \hline
		\textbf{Superclass} & \texttt{ViewAlert\_Controller}
		\\ \hline
		% Sostituire ad 1 il numero di classi figlie
		\multirow{1}{*}{\textbf{Subclasses}} & Nessuna
		\\ \hline
		\textbf{Responsability} & Aggiunta di una sessione alla conferenza
		\\ \hline
		\multirow{6}{*}{\textbf{Collaborations}} & \texttt{Conferenza} \\
		& \texttt{Sessione}\\
		& \texttt{Utente}\\
		& \texttt{Organizzatore}\\
		& \texttt{Sala}\\
		& \texttt{VisualizzaSessioniConferenza\_Controller}		\\ \hline
	\end{tabular}
\end{table}

\begin{table}[h!]
	\begin{tabular}{|l|l|}
		\hline 
		\textbf{Class name} & \texttt{ViewProgrammaConferenza\_Controller}
		\\ \hline
		\textbf{Superclass} & \texttt{ViewAlert\_Controller}
		\\ \hline
		% Sostituire ad 1 il numero di classi figlie
		\multirow{1}{*}{\textbf{Subclasses}} & Nessuna
		\\ \hline
		\textbf{Responsability} & Gestisce gli eventi che compongono il programma e fornisce diverse funzionalità di navigazione
		\\ \hline
		\multirow{17}{*}{\textbf{Collaborations}} & \texttt{Conferenza} \\
		& \texttt{Programma}\\
		& \texttt{Utente}\\
		& \texttt{Sessione}\\
		& \texttt{ActivityModel}\\
		& \texttt{Intervento} \\
		& \texttt{Intervallo} \\
		& \texttt{EventoSociale} \\
		& \texttt{Speaker} \\
		& \texttt{AddEventoSociale\_Controller} \\
		& \texttt{ViewProgramma\_Controller} \\
		& \texttt{AddIntervallo\_Controller} \\
		& \texttt{AddIntervento\_Controller} \\
		& \texttt{ChooseKeynote\_Controller} \\
		& \texttt{ShowInfoEventoSociale\_Controller} \\
		& \texttt{ShowInfoIntervallo\_Controller} \\
		& \texttt{ShowInfoIntervento\_Controller} \\ \hline
	\end{tabular}
\end{table}


\begin{table}[h!]
	\begin{tabular}{|l|l|}
		\hline 
		\textbf{Class name} & \texttt{AddIntervallo\_Controller}
		\\ \hline
		\textbf{Superclass} & \texttt{ViewAlert\_Controller}
		\\ \hline
		% Sostituire ad 1 il numero di classi figlie
		\multirow{1}{*}{\textbf{Subclasses}} & Nessuna
		\\ \hline
		\textbf{Responsability} & Aggiunta di un Intervallo al Prgogramma
		\\ \hline
		\multirow{3}{*}{\textbf{Collaborations}} & \texttt{Programma }\\
		& \texttt{Intervallo}\\
		& \texttt{ViewProgrammaConferenza\_Controller}
		\\ \hline
	\end{tabular}
\end{table}

\begin{table}[h!]
	\begin{tabular}{|l|l|}
		\hline 
		\textbf{Class name} & \texttt{AddEventoSociale\_Controller}
		\\ \hline
		\textbf{Superclass} & \texttt{ViewAlert\_Controller}
		\\ \hline
		% Sostituire ad 1 il numero di classi figlie
		\multirow{1}{*}{\textbf{Subclasses}} & Nessuna
		\\ \hline
		\textbf{Responsability} & Aggiunta di un EventoSociale al Prgogramma
		\\ \hline
		\multirow{3}{*}{\textbf{Collaborations}} & \texttt{Programma }\\
		& \texttt{EventoSociale}\\
		& \texttt{ViewProgrammaConferenza\_Controller}
		\\ \hline
	\end{tabular}
\end{table}

\begin{table}[h!]
	\begin{tabular}{|l|l|}
		\hline 
		\textbf{Class name} & \texttt{AddIntervento\_Controller}
		\\ \hline
		\textbf{Superclass} & \texttt{ViewAlert\_Controller}
		\\ \hline
		% Sostituire ad 1 il numero di classi figlie
		\multirow{1}{*}{\textbf{Subclasses}} & Nessuna
		\\ \hline
		\textbf{Responsability} & Aggiunta di un Intervento al Prgogramma
		\\ \hline
		\multirow{3}{*}{\textbf{Collaborations}} & \texttt{Programma} \\
		& \texttt{Intervento}\\
		& \texttt{Speakers}\\
		& \texttt{AddSpeaker\_Controller}\\
		& \texttt{ViewProgrammaConferenza\_Controller}
		\\ \hline
	\end{tabular}
\end{table}

\begin{table}[h!]
	\begin{tabular}{|l|l|}
		\hline 
		\textbf{Class name} & \texttt{AddSpeaker\_Controller}
		\\ \hline
		\textbf{Superclass} & \texttt{ViewAlert\_Controller}
		\\ \hline
		% Sostituire ad 1 il numero di classi figlie
		\multirow{1}{*}{\textbf{Subclasses}} & Nessuna
		\\ \hline
		\textbf{Responsability} & Aggiunta di uno Speaker alla lista degli Speaker disponibili
		\\ \hline
		\multirow{2}{*}{\textbf{Collaborations}} & \texttt{Speakers} \\
		& \texttt{Enti}
		\\ \hline
	\end{tabular}
\end{table}
\clearpage
\pagebreak

\subsubsection{Controller.Login}
Symposium ha un sistema di Registrazione e di Login utente, in modo che ogni singolo utente possa modificare esclusivamente le conferenze di cui è proprietario.
\begin{table}[h!]
	\begin{tabular}{|l|l|}
		\hline 
		\textbf{Class name} & \texttt{Login\_Controller}
		\\ \hline
		\textbf{Superclass} & Nessuna
		\\ \hline
		% Sostituire ad 1 il numero di classi figlie
		\multirow{1}{*}{\textbf{Subclasses}} & Nessuna
		\\ \hline
		\textbf{Responsability} & Schermata di Login
		\\ \hline
		\multirow{3}{*}{\textbf{Collaborations}} & \texttt{Utente} \\
		& \texttt{Landing\_Controller}\\
		& \texttt{Register\_Controller}
		\\ \hline
	\end{tabular}
\qquad
	\begin{tabular}{|l|l|}
		\hline 
		\textbf{Class name} & \texttt{Register\_Controller}
		\\ \hline
		\textbf{Superclass} & Nessuna
		\\ \hline
		% Sostituire ad 1 il numero di classi figlie
		\multirow{1}{*}{\textbf{Subclasses}} & Nessuna
		\\ \hline
		\textbf{Responsability} & Schermata di registrazione Utente
		\\ \hline
		\multirow{3}{*}{\textbf{Collaborations}} & \texttt{Utente} \\
		& \texttt{Enti}\\
		& \texttt{Login\_Controller}
		\\ \hline
	\end{tabular}
\end{table}  


\subsubsection{Controller.Stats}
Una delle feature di Symposium è la visualizzazione delle statistiche. Il sistema infatti fornisce all'utente un resoconto delle conferenze organizzate, delle istituzioni ed aziende coinvolte nonché un calcolo su base mensile ed annuale del tasso di appartenenza dei vari speaker che partecipano alle varie conferenze.
\begin{table}[h!]
	\begin{tabular}{|l|l|}
		\hline 
		\textbf{Class name} & \texttt{MonthlyStatWindow\_Controller}
		\\ \hline
		\textbf{Superclass} & \texttt{ViewAlert\_Controller}
		\\ \hline
		% Sostituire ad 1 il numero di classi figlie
		\multirow{1}{*}{\textbf{Subclasses}} & Nessuna
		\\ \hline
		\textbf{Responsability} & Calcolo delle statistiche su base mensile
		\\ \hline
		\multirow{1}{*}{\textbf{Collaborations}} & \texttt{Stats} \\ \hline
	\end{tabular}
\end{table}
\begin{table}[h!]
	\begin{tabular}{|l|l|}
		\hline 
		\textbf{Class name} & \texttt{YearlyStatWindow\_Controller}
		\\ \hline
		\textbf{Superclass} & \texttt{ViewAlert\_Controller}
		\\ \hline
		% Sostituire ad 1 il numero di classi figlie
		\multirow{1}{*}{\textbf{Subclasses}} & Nessuna
		\\ \hline
		\textbf{Responsability} & Calcolo delle statistiche su base annuale
		\\ \hline
		\multirow{1}{*}{\textbf{Collaborations}} & \texttt{Stats }\\ \hline
	\end{tabular}
\end{table}
\begin{table}[h!]
	\begin{tabular}{|l|l|}
		\hline 
		\textbf{Class name} & \texttt{VisualizzaStatistiche\_Controller}
		\\ \hline
		\textbf{Superclass} & Nessuna
		\\ \hline
		% Sostituire ad 1 il numero di classi figlie
		\multirow{1}{*}{\textbf{Subclasses}} & Nessuna
		\\ \hline
		\textbf{Responsability} & Finestra principale per la visualizzazione delle statistiche di sistema
		\\ \hline
		\multirow{2}{*}{\textbf{Collaborations}} & \texttt{YearlyStatWindow\_Controller} \\
		& \texttt{MonthlyStatWindow\_Controller} \\ \hline
	\end{tabular}
\end{table}

\subsubsection{Controller.View}
I controller presenti nel package \texttt{Controller.View} gestiscono il funzionamento delle classi il cui unico scopo è visualizzare i dettagli senza ricevere input dall'Utente.
\begin{table}[h!]
	\begin{tabular}{|l|l|}
		\hline 
		\textbf{Class name} & \texttt{ViewAlert\_Controller}
		\\ \hline
		\textbf{Superclass} & Nessuna
		\\ \hline
		% Sostituire ad 1 il numero di classi figlie
		\multirow{1}{*}{\textbf{Subclasses}} & Tutti i controller
		\\ \hline
		\textbf{Responsability} & Visualizzare i dettagli di un Evento Sociale
		\\ \hline
		\multirow{1}{*}{\textbf{Collaborations}} & Nessuna \\ \hline
	\end{tabular}
\end{table} 
\begin{table}[h!]
	\begin{tabular}{|l|l|}
		\hline 
		\textbf{Class name} & \texttt{ShowInfoEventoSociale\_Controller}
		\\ \hline
		\textbf{Superclass} & Nessuna
		\\ \hline
		% Sostituire ad 1 il numero di classi figlie
		\multirow{1}{*}{\textbf{Subclasses}} & Nessuna
		\\ \hline
		\textbf{Responsability} & Visualizzare i dettagli di un Evento Sociale
		\\ \hline
		\multirow{2}{*}{\textbf{Collaborations}} & EventoSociale\\
		& \texttt{ActivityModel}
		\\ \hline
	\end{tabular}
\end{table} 

\begin{table}[h!]
	\begin{tabular}{|l|l|}
		\hline 
		\textbf{Class name} & \texttt{ShowInfoIntervento\_Controller}
		\\ \hline
		\textbf{Superclass} & Nessuna
		\\ \hline
		% Sostituire ad 1 il numero di classi figlie
		\multirow{1}{*}{\textbf{Subclasses}} & Nessuna
		\\ \hline
		\textbf{Responsability} & Visualizzare i dettagli di un Intervento
		\\ \hline
		\multirow{2}{*}{\textbf{Collaborations}} & \texttt{Intervento}\\
		& \texttt{ActivityModel}
		\\ \hline
	\end{tabular}
\end{table} 

\begin{table}[h!]
	\begin{tabular}{|l|l|}
		\hline 
		\textbf{Class name} &\texttt{ ShowInfoIntervallo\_Controller}
		\\ \hline
		\textbf{Superclass} & Nessuna
		\\ \hline
		% Sostituire ad 1 il numero di classi figlie
		\multirow{1}{*}{\textbf{Subclasses}} & Nessuna
		\\ \hline
		\textbf{Responsability} & Visualizzare i dettagli di un Intervallo
		\\ \hline
		\multirow{2}{*}{\textbf{Collaborations}} & \texttt{Intervallo}\\
		& \texttt{ActivityModel}
		\\ \hline
	\end{tabular}
\end{table} 

\begin{table}[h!]
	\begin{tabular}{|l|l|}
		\hline 
		\textbf{Class name} & \texttt{VisualizzaConferenza\_Controller}
		\\ \hline
		\textbf{Superclass} & \texttt{ViewAlert\_Controller}
		\\ \hline
		% Sostituire ad 1 il numero di classi figlie
		\multirow{1}{*}{\textbf{Subclasses}} & Nessuna
		\\ \hline
		\textbf{Responsability} & Visualizzare i dettagli di una conferenza
		\\ \hline
		\multirow{5}{*}{\textbf{Collaborations}} & \texttt{Conferenza}\\
		& \texttt{Comitato}\\
		& \texttt{Ente} \\
		& \texttt{Sessione} \\
		& \texttt{Sponsorizzazione}
		\\ \hline
	\end{tabular}
\end{table} 

\begin{table}[h!]
	\begin{tabular}{|l|l|}
		\hline 
		\textbf{Class name} & \texttt{VisualizzaConferenze\_Controller}
		\\ \hline
		\textbf{Superclass} & \texttt{ViewAlert\_Controller}
		\\ \hline
		% Sostituire ad 1 il numero di classi figlie
		\multirow{1}{*}{\textbf{Subclasses}} & Nessuna
		\\ \hline
		\textbf{Responsability} & Visualizzare un elenco di conferenze in base a criteri specifici
		\\ \hline
		\multirow{3}{*}{\textbf{Collaborations}} &\texttt{Conferenze}\\
		& \texttt{Sedi}\\
		& \texttt{Conferenza}
		\\ \hline
	\end{tabular}
\end{table} 

\begin{table}[h!]
	\begin{tabular}{|l|l|}
		\hline 
		\textbf{Class name} & \texttt{VisualizzaSessione\_Controller}
		\\ \hline
		\textbf{Superclass} & \texttt{ViewAlert\_Controller}
		\\ \hline
		% Sostituire ad 1 il numero di classi figlie
		\multirow{1}{*}{\textbf{Subclasses}} & Nessuna
		\\ \hline
		\textbf{Responsability} & Visualizzare i dettagli di una sessione
		\\ \hline
		\multirow{2}{*}{\textbf{Collaborations}} & \texttt{Sessione}\\
		& \texttt{Programma}
		\\ \hline
	\end{tabular}
\end{table} 