\section{Suddivisione delle classi}
Nell'organizzazione del codice sorgente del nostro progetto abbiamo applicato la seguente suddivisione in packages per le classi, catalogandole seguendo l'euristica Model-View-Controller:
\begin{enumerate}
\item \texttt{Model}
\begin{enumerate}
\item \texttt{DAO}
\item \texttt{DbConfig}
\item \texttt{Entities}
\item \texttt{Utilities}
\end{enumerate}
\item \texttt{View}
\begin{enumerate}
\item \texttt{FXML}
\item \texttt{CSS}
\end{enumerate}
\item \texttt{Controller}
\item \texttt{Exceptions}
\end{enumerate}

All'interno del package \texttt{View} sono presenti tutti i file \texttt{.fxml} che descrivono le interfacce grafiche dell'applicazione e i file \texttt{.css} utilizzati per la loro personalizzazione, mentre all'interno del package \texttt{Controller} sono presenti le classi che implementano i controller delle interfacce grafiche.
\bigskip

All'interno del package \texttt{Model} sono presenti vari packages che implementano il modello del nostro progetto come presentato nel Diagramma \ref{uml:modellodominio}. Nel sub-package \texttt{DAO} sono state inserite tutte le classi utilizzate per implementare il pattern DAO, mentre nel sub-package \texttt{DbConfig} è presente la classe per la configurazione del database. Infine,  nel sub-package \texttt{Utilities} sono presenti le classi che ci sono state necessarie per la gestione di molteplici istanze delle classi di dominio.