\section{Definizione del problema}
Si sviluppi un sistema informativo, composto da una base di dati relazionale e da un applicativo Java dotato di GUI (Swing o JavaFX), per la gestione di \textbf{conferenze scientifiche}. 
\bigskip

Ogni conferenza ha una data di inizio e di fine, una collocazione (sede, indirizzo), uno o più enti che la organizzano, degli sponsor (che coprono in parte le spese), una descrizione, ed un gruppo di organizzatori, che può essere distinto in comitato scientifico e comitato locale (che si occupa cioè della logistica). Di ognuno degli organizzatori, così come di tutti i partecipanti, si riportano titolo, nome, cognome, email ed istituzione di afferenza. 
\bigskip

Ogni conferenza può avere una o più sessioni, anche in parallelo fra loro. Ogni sessione ha una locazione all'interno della sede. Per ogni
sessione c'è un programma, che prevede la presenza di un coordinatore (chair) che gestisce la sessione, ed eventualmente di un keynote speaker (un partecipante di particolare rilievo invitato dagli organizzatori). Ogni sessione avrà quindi una successione di interventi ad orari predefiniti e di specifici partecipanti. Per ogni intervento si conserva un abstract (un breve testo in cui viene spiegato il contenuto del lavoro presentato).
\bigskip

Si deve poter considerare la presenza di spazi di intervallo (coffee breaks, pranzo) ma anche la presenza di eventi sociali (cene, gite, etc).