\section{Symposium: un applicativo per la gestione di conferenze scientifiche}
\textit{Symposium} è un applicativo per la gestione di conferenze scientifiche, sviluppato in JavaFX e basato su PostgreSQL15. Il sistema è progettato per fornire una piattaforma completa e intuitiva per gli organizzatori delle conferenze, consentendo loro di pianificare, gestire e vsualizzare le varie attività svolte durante le conferenze.

\subsection{Caratteristiche principali di Symposium}
All'interno di \textit{Symposium} è possibile:
\begin{enumerate}
\item \textbf{Creare nuove conferenze:}
	Ogni conferenza viene registrata nel sistema con dettagli come la data di inizio e fine, la collocazione (sede e indirizzo), gli enti organizzatori, gli sponsor coinvolti e ogni eventuale sessione prevista durante la conferenza.  Una descrizione dell'evento sarà disponibile per fornire informazioni generali.
\item \textbf{Gestione dei comitati:}
	Per ogni conferenza, sono registrati i dettagli dei membri del comitato scientifico e del comitato locale, che si occupano rispettivamente degli aspetti scientifici e logistici dell'evento.
\item \textbf{Gestione della conferenza:}
	Per ogni conferenza è possibile modificare i suoi dettagli generali (quali il titolo, l'inizio, la fine e la sede), modificare gli enti organizzatori e le varie sponsorizzazioni oppure slittare la conferenza.
\item \textbf{Gestione delle sessioni:}
	Per ogni sessione, è possibile creare un programma dettagliato con gli orari dei vari punti in programma.
\end{enumerate}


\subsection{Interfaccia Utente}

L'interfaccia utente di Symposium sarà realizzata utilizzando JavaFX, fornendo un'esperienza utente intuitiva e piacevole. Gli organizzatori possono accedere alla piattaforma per registrarsi e visualizzare i dettagli delle conferenze già presenti nel database. Ogni utente avrà la possibilità di gestire le proprie conferenze, specificare il programma delle sessioni, specificare la nomina dei comitati e altro ancora.
\bigskip

In conclusione, Symposium è un sistema informativo embrionale per la gestione di conferenze scientifiche che offre funzionalità complete e una piattaforma intuitiva per organizzatori.