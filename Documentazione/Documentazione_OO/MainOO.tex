\documentclass[a4paper, oneside, 10pt]{book}
\usepackage{graphicx}
\usepackage{float}
\usepackage{adjustbox}
\usepackage{listings}
\usepackage{setspace}
\usepackage[labelfont={bf},textfont=it]{caption}
\usepackage{fullpage}
\usepackage{booktabs}
\usepackage{subfig}
\usepackage{enumerate}
\usepackage{pgfplots}
\pgfplotsset{compat=1.8}
\usepackage{tabularx}
\usepackage{multirow}
\usepackage[T1]{fontenc}
\usepackage{inputenc}
\usepackage{hyperref}
\usepackage[italian]{babel}
\usepackage{longtable}
\usepackage{import} 
\usepackage{mathtools}
\usepackage{amssymb,amsmath,amsthm}
\setlength{\parindent}{0pt}
	\setcounter{secnumdepth}{3}
{\renewcommand{\arraystretch}{1.2}%
\begin{document}
\pagestyle{empty}
\begin{titlepage}
	\begin{center}
		\setstretch{1.2}
		\setlength{\parskip}{2ex}
		
		
		\Large\textsc{Università degli Studi di Napoli Federico II}
		
		\includegraphics[width=3cm]{Immagini/logo-federico-II.pdf}
		
		\Large\textsc{Scuola Politecnica e delle Scienze di Base}
		
		\large\textsc{Dipartimento di Ingegneria Elettrica e Tecnologie dell'Informazione}
		
		\large\textsc{Corso di Laurea Triennale in Informatica}
		
		\textsc{Progetto d'esame di Object Orientation}
		\vfill
		\setstretch{2}
		\huge\textsc{Progettazione e sviluppo di un applicativo in Java per la gestione di conferenze scientifiche}
		\vfill
		\setstretch{1}
		\begin{minipage}[t]{.49\textwidth}
			\large
			
			\textbf{Relatore}\par
			Professore Sergio \textsc{Di Martino}
		\end{minipage}\hfill
	\begin{minipage}[t]{.45\textwidth}
		\large
		\hspace{3.3cm}\textbf{Candidati}\par
		\hfill\begin{tabular}{l}
			 Antonio \textsc{Caporaso} \\ 
			 matr: \texttt{N86003458} \\
			 Giorgio \textsc{Di Fusco} \\
			  matr: \texttt{N86004389} \\
		 \end{tabular}
	\end{minipage}
\vfill
		
		\large Anno Accademico 2022-2023
	\end{center}
\end{titlepage}

\newpage
\textit{Questa pagina è stata lasciata intenzionalmente vuota.}
\newpage
\tableofcontents
\listoffigures
\listoftables

\chapter{Introduzione}
	\import{Argomenti/1-Introduzione}{Traccia.tex}
	\import{Argomenti/1-Introduzione}{Descrizione.tex}
\chapter{Progettazione del software: CRC card e class diagrams}
	\import{Argomenti/2-Progettazione}{Entities.tex}
	\import{Argomenti/2-Progettazione}{Packages.tex}
	\import{Argomenti/2-Progettazione}{Model.tex}
	\import{Argomenti/2-Progettazione}{Controllers.tex}
	\import{Argomenti/2-Progettazione}{ClassDiagram.tex}
\chapter{Sequence Diagram e Mockup}
	\import{Argomenti/3-Diagrammi}{Sequence.tex}
	\import{Argomenti/3-Diagrammi}{Mockup.tex}
\end{document}
