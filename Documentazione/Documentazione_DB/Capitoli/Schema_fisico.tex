\section{Implementazione fisica}
\subsection{Definizione delle tabelle}
\begin{lstlisting}[language=SQL, style=mystyle]
	CREATE SCHEMA conference;
	set search_path to conference;
	
	--Definizione tipi
	create type comitato_st as enum ('locale','scientifico');
	create type intervallo_st as enum ('pranzo','coffee break');
	
	--Tabella delle istituzioni
	create table ente(
	id_ente serial primary key,
	nome text not null unique,
	sigla varchar(7) not null,
	unique (nome,sigla)
	);
	
	--Tabella delle sedi
	create table sede(
	id_serial serial primary key,
	nome text not null,
	via text not null,
	civico varchar(5) not null,
	cap varchar(5) not null,
	city text not null
	);
	
	-- Tabella degli sponsor
	create table sponsor(
	id_sponsor serial primary key,
	nome text not null
	);

	-- Tabella dei comitati
	create table comitato(
	id_comitato serial primary key,
	tipologia comitato_st not null
	);
	
	-- Tabella degli organizzatori
	create table organizzatore(
	id_organizzatore serial primary key,
	nome text not null,
	cognome text not null,
	titolo varchar(10) not null,
	email text not null unique,
	id_ente integer references ente(id_ente) on delete cascade
	);
	
	-- Tabella delle sale
	create table sala(
	id_sala serial primary key,
	nome text not null,
	capienza integer not null,
	id_sede integer references sede(id_serial) on delete cascade
	);
	
	-- Tabella delle conferenze
	create table conferenza(
	id_conferenza serial primary key,
	titolo text not null,
	descrizione text not null,
	inizio timestamp not null,
	fine timestamp not null,
	id_sede integer references sede(id_serial) on delete set null,
	id_comitato_scientifico integer references comitato(id_comitato) on delete set null,
	id_comitato_locale integer references comitato(id_comitato) on delete set null,
	check (inizio <= fine),
	check (inizio >= now())
	);
	
	-- Tabella dei partecipanti
	create table partecipante(
	id_partecipante serial primary key,
	nome text not null,
	cognome text not null,
	titolo varchar(10) not null,
	email text not null unique,
	id_ente integer references ente(id_ente) on delete set null
	);
	
	-- Tabella delle sessioni
	create table sessione(
	id_sessione serial primary key,
	titolo text not null,
	inizio timestamp not null,
	fine timestamp not null,
	id_coordinatore integer references organizzatore(id_organizzatore) on delete set null,
	id_conferenza integer references conferenza(id_conferenza) on delete cascade,
	id_sala integer references sala(id_sala) on delete set null,
	check (inizio <= fine)
	);
	
	--Tabella ponte tra la tabella Sessione e la tabella Partecipante
	create table PartecipanteSessione(
	id_partecipante integer references partecipante(id_partecipante) on delete cascade,
	id_sessione integer references sessione(id_sessione) on delete cascade,
	unique (id_partecipante,id_sessione)
	);
	
	-- Tabella ponte tra le tabelle Ente e la tabella Conferenza
	create table ente_conferenza(
	id_ente integer references ente(id_ente) on delete cascade,
	id_conferenza integer references conferenza(id_conferenza) on delete cascade,
	unique (id_ente,id_conferenza)
	);
	
	-- Tabella per la rappresentazione delle valute
	create table valuta(
	iso char(3) primary key,
	nome text not null,
	simbolo text not null
	);
	
	--Tabella ponte tra Sponsor e Conferenza
	create table SponsorConferenza(
	id_sponsor integer references sponsor(id_sponsor) on delete cascade,
	contributo numeric(1000,2) not null,
	valuta char(3) references valuta(iso) not null,
	id_conferenza integer references conferenza(id_conferenza) on delete cascade,
	unique (id_sponsor,id_conferenza)
	);
	
	-- Tabella per gli speaker delle sessioni
	create table speaker(
	id_speaker serial primary key,
	nome text not null,
	cognome text not null,
	titolo varchar(10) not null,
	email text not null unique,
	id_ente integer references ente(id_ente) on delete set null
	);
	
	-- Tabella per i programmi delle sessioni
	create table programma(
	id_programma serial primary key,
	id_sessione integer references sessione(id_sessione) on delete cascade,
	id_keynote integer references speaker(id_speaker) on delete set null,
	unique (id_programma, id_sessione)
	);
	
	-- Tabella per gli interventi in programma
	create table intervento(
	id_intervento serial primary key,
	titolo text not null,
	abstract text not null,
	inizio timestamp not null,
	fine timestamp not null,
	id_speaker integer references speaker(id_speaker) on delete cascade,
	id_programma integer references programma(id_programma) on delete cascade,
	unique (id_speaker,id_programma),
	check (inizio <= fine)
	);
	
	-- Tabella per gli intervalli in programma
	create table intervallo(
	id_intervallo serial primary key,
	tipologia intervallo_st not null,
	inizio timestamp not null,
	fine timestamp not null,
	check (inizio <= fine),
	id_programma integer references programma(id_programma) on delete cascade
	);
	
	-- Tabella per gli eventi in programma
	create table evento(
	id_evento serial primary key,
	tipologia text not null,
	inizio timestamp not null,
	fine timestamp not null,
	check (inizio <= fine),
	id_programma integer references programma(id_programma) on delete cascade
	);
	
	-- Tabella ponte tra Organizzatore e Comitato
	create table organizzatore_comitato(
	id_organizzatore_comitato serial primary key,
	id_organizzatore integer references organizzatore(id_organizzatore) on delete cascade,
	id_comitato integer references comitato(id_comitato) on delete cascade,
	unique (id_organizzatore,id_comitato)
	);
	
\end{lstlisting}
\subsection{Popolamento}
\subsection{Trigger}
\subsection{Funzioni}
\subsection{Procedure}