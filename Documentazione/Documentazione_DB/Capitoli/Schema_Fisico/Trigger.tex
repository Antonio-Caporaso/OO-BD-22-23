\section{Definizione dei trigger}
\subsection{\texttt{Check\_Programma}}\label{vincolo:check_programma}
Stando al vincolo interrelazionale \texttt{check\_programma} in un programma non devono esserci eventi, intervalli o interventi che si sovrappongono. Per questo motivo definiamo il trigger \texttt{check\_programma} che viene eseguito per ogni inserimento o aggiornamento nelle tabelle \textsc{Intervento}, \textsc{Intervallo} ed \textsc{Evento}.
\begin{lstlisting}[language=SQL,style=mystyle,caption={check\_programma\_entry}]
create or replace function check_programma() returns trigger as $$
declare
    inizio_evento timestamp;
    fine_evento timestamp;
    inizio_intervallo timestamp;
    fine_intervallo timestamp;
    inizio_intervento timestamp;
    fine_intervento timestamp;
    intervento_id integer;
    intervallo_id integer;
    evento_id integer;
    interventi_cur cursor for select id_intervento from intervento where id_programma = new.id_programma;
    intervalli_cur cursor for select id_intervallo from intervallo where id_programma = new.id_programma;
    eventi_cur cursor for select id_evento from evento where id_programma = new.id_programma;
begin
open interventi_cur;
loop 
    fetch interventi_cur into intervento_id;
    exit when not found;
    select inizio,fine into inizio_intervento,fine_intervento
    from intervento
    where id_intervento = intervento_id;
    if (new.inizio>=inizio_intervento AND new.fine<=fine_intervento) then
        raise exception 'Impossibile inserire questo intervento in questo orario';
    end if;
end loop;
close interventi_cur;
open intervalli_cur;
loop 
    fetch intervalli_cur into intervallo_id;
    exit when not found;
    select inizio,fine into inizio_intervallo,fine_intervallo
    from intervallo
    where id_intervallo = intervallo_id;
    if (new.inizio>=inizio_intervallo AND new.fine<=fine_intervallo) then
        raise exception 'Impossibile inserire questo intervallo in questo orario';
    end if;
end loop;
close intervalli_cur;
open eventi_cur;
loop 
    fetch eventi_cur into evento_id;
    exit when not found;
    select inizio,fine into inizio_evento,fine_evento
    from evento
    where id_evento = evento_id;
    if (new.inizio>=inizio_evento AND new.fine<=fine_evento) then
        raise exception 'Impossibile inserire questo evento in questo orario';
    end if;
end loop;
close eventi_cur;
return new;
end;
$$ language plpgsql;

create trigger check_programma
before insert or update on intervento
for each row
execute function check_programma();

create trigger check_programma
before insert or update on intervallo
for each row
execute function check_programma();

create trigger check_programma
before insert or update on evento
for each row
execute function check_programma();
\end{lstlisting}

\subsection{\texttt{Check\_Data\_Intervento, Check\_Data\_Intervallo, Check\_Data\_Evento}}\label{trigger:check_data}
Ogni volta che viene inserito o aggiornato un intervento, un intervallo o un evento bisogna controllare sempre che le loro date di inizio e di fine siano coerenti con quella della sessione cui appartengono:
\begin{lstlisting}[caption={check\_data\_intervento},language=SQL,style=mystyle]
create or replace function check_data() returns trigger as $$
declare
    inizio_sessione timestamp;
    fine_sessione timestamp;
begin
    select inizio,fine into inizio_sessione,fine_sessione
    from sessione
    where id_sessione = (select id_sessione from programma where id_programma = new.id_programma);

    if (new.inizio < inizio_sessione OR new.fine > fine_sessione) then
        raise exception 'L''intervento non e'' compreso nella sessione';
    end if;
    return new;
end;
$$ language plpgsql;

create trigger check_data_evento
before insert or update on evento
for each row
execute function check_data();

create trigger check_data_intervento
before insert or update on intervento
for each row
execute function check_data();

create trigger check_data_intervallo
before insert or update on intervallo
for each row
execute function check_data();
\end{lstlisting}
\subsection{\texttt{Create\_Programma\_Sessione}}
Il trigger \texttt{Create\_Programma\_Sessione} viene attivato subito dopo aver inserito una nuova sessione ed effettua l'inserimento di un programma vuoto associato alla sessione.
\begin{lstlisting}[language=SQL, caption={\texttt{create\_programma\_sessione}},style=mystyle]
create or replace function create_programma_sessione() returns trigger as $$
begin
    insert into programma(id_sessione) values (new.id_sessione);
    return new;
end;
$$ language plpgsql;

create trigger create_programma_sessione
after insert on sessione
for each row
execute function create_programma_sessione();
\end{lstlisting}
\subsection{\texttt{Check\_Sala\_In\_Sede}}\label{trigger:salainsede}
Quando inseriamo una sessione bisogna stare attenti che la chiave esterna della sala sia effettivamente una sala appartenente alla sede che ospita la conferenza della sessione in questione. Il trigger \texttt{check\_sala\_sessione} effettua quindi questo controllo prima di ciascun inserimento nella tabella \textsc{Sessione}:
\begin{lstlisting}[language=SQL, style=mystyle]
create or replace function check_sala_sessione() returns trigger as $$
declare
    sede integer;
    sala integer;
begin
    select id_sede into sede
    from conferenza
    where id_conferenza = new.id_conferenza;
    
    select id_sala into sala
    from sala
    where id_sala = new.id_sala;
    
    IF sala IS NULL THEN
        Return new;
    END IF;
    
    IF sala NOT IN (
        SELECT id_sala
        FROM sala
        WHERE id_sede = sede
    ) THEN
        RAISE exception 'La sala selezionata non appartiene alla sede della conferenza';
    END IF;
    
    return new;
end;
$$ language plpgsql;


create trigger check_sala_sessione
before insert or update on sessione
for each row
execute function check_sala_sessione();
\end{lstlisting}
\subsection{\texttt{Check\_Data\_Sessione}}\label{trigger:datasessione}
Analogamente ai trigger \ref{trigger:check_data} si definisce il trigger \texttt{check\_data\_sessione} che controlla che l'inizio e la fine di ciascuna sessione siano coerenti con quelle della relativa conferenza:
\begin{lstlisting}[language=SQL, style=mystyle, caption={\texttt{check\_data\_sessione}}]
create or replace function check_data_sessione() returns trigger as $$
declare
    inizio_conferenza timestamp;
    fine_conferenza timestamp;
begin
    select inizio, fine into inizio_conferenza,fine_conferenza
    from conferenza
    where id_conferenza = new.id_conferenza;

    if (new.inizio < inizio_conferenza OR new.fine > fine_conferenza) then
        raise exception 'La sessione non e'' compresa nella conferenza';
    end if;
    return new;
end;
$$ language plpgsql;

create trigger check_data_sessione
before insert or update on sessione
for each row
execute function check_data_sessione();
\end{lstlisting}
\subsection{\texttt{Check\_Coordinatore\_Sessione}}\label{trigger:check_coordinatore}
Quando si specifica il coordinatore della sessione bisogna controllare che questi appartenga al comitato scientifico che è il gruppo di organizzatori che si occupa della gestione delle conferenze e delle sessioni:
\begin{lstlisting}[caption={check\_coordinatore\_sessione}, language=sql, style=mystyle]
create or replace function check_coordinatore_sessione() returns trigger as $$
declare 
    id_comitato_scientifico_conferenza integer;
begin

select comitato_s into id_comitato_scientifico_conferenza
from conferenza c
where c.id_conferenza = new.id_conferenza;

if (new.id_coordinatore is not null) then
    if (id_comitato_scientifico_conferenza not in (select id_comitato from organizzatore_comitato where id_organizzatore = new.id_coordinatore))  then
        raise exception 'Il coordinatore della sessione deve appartenere al comitato scientifico della conferenza';
    end if;
end if;
return new;
end;
$$ language plpgsql;

create trigger check_coordinatore_sessione
before insert or update on sessione
for each row
execute function check_coordinatore_sessione();
\end{lstlisting}
\subsection{\texttt{Create\_Comitati\_Conferenza}}
Gli enti che organizzano le conferenze nominano due comitati per ogni conferenza che organizzano. Per questo motivo, ogni volta che viene inserita una nuova conferenza viene attivato il trigger \texttt{create\_comitati\_conferenza} che si occupa di creare due nuovi comitati di tipologica \textit{scientifica} e \textit{locale} e associarli alla nuova conferenza appena create:
\begin{lstlisting}[language=SQL, style=mystyle, caption={\texttt{create\_comitati\_conferenza}}]
create or replace function create_comitati_conferenza() returns trigger as $$
declare 
    id_comitatoscientifico integer;
    id_comitatolocale integer;
begin
    insert into comitato(tipologia) values ('scientifico') returning id_comitato into id_comitatoscientifico;
    insert into comitato(tipologia) values ('locale') returning id_comitato into id_comitatolocale;
    update conferenza set comitato_s = id_comitatoscientifico, comitato_l = id_comitatolocale where id_conferenza = new.id_conferenza;
    return new;
end;
$$ language plpgsql;

create trigger create_comitati_conferenza
after insert on conferenza
for each row
execute function create_comitati_conferenza();
\end{lstlisting}
\subsection{\texttt{Check\_Comitati\_Conferenza}}\label{trigger:check_comitati}
Ogni volta che si aggiorna una conferenza bisogna controllare che le chiavi esterne dei due comitati si riferiscano sempre a comitati della tipologia richiesta:
\begin{lstlisting}[language=SQL, caption={\texttt{check\_comitati\_conferenza}},style=mystyle]
create or replace function check_comitati_conferenza() returns trigger as $$
declare 
    id_comitato_scientifico integer;
    id_comitato_locale integer;
begin
    select id_comitato into id_comitato_scientifico
    from comitato
    where id_comitato = new.comitato_s;
    
    select id_comitato into id_comitato_locale
    from comitato
    where id_comitato = new.comitato_l;
    
    IF id_comitato_scientifico IS NULL THEN
        return new;
    END IF;
    
    IF id_comitato_locale IS NULL THEN
        return new;
    END IF;
    
    IF (select tipologia from comitato where id_comitato = id_comitato_scientifico) <> 'scientifico' THEN
        RAISE EXCEPTION 'Il comitato scientifico deve essere scientifico';
    END IF;
    
    IF (select tipologia from comitato where id_comitato = id_comitato_locale) <> 'locale' THEN
        RAISE EXCEPTION 'Il comitato locale deve essere locale';
    END IF;
    
    return new;
end;
$$ language plpgsql;

create trigger check_comitati_conferenza
before update on conferenza
for each row
execute function check_comitati_conferenza();
\end{lstlisting}
\subsection{\texttt{Check\_Sede\_Disponibile}}\label{trigger:check_sede_disponibile}
Ogni volta che si aggiunge una nuova conferenza bisogna prima verificare la disponibilità della sede. Una sede risulta disponibile ha almeno una sala libera da sessioni nel periodo indicato.
\begin{lstlisting}[caption={\texttt{Check\_sede\_libera}}, language=SQL, style=mystyle]
create or replace function check_sede_libera() returns trigger as $$
begin
    if (new.id_sede is null) then
        return new;
    end if;

    if (new.inizio is null) then
        return new;
    end if;

    if (new.fine is null) then
        return new;
    end if;

    if ( select count(*) 
         from sala 
         where id_sede = new.id_sede 
            and id_sala not in 
                (select id_sala 
                from sessione 
                where id_conferenza = new.id_conferenza 
                and (inizio < new.fine and fine > new.inizio)
                )
        ) = 0 then
        raise exception 'La sede non ha sale libere';
    end if;
    return new;
end;
$$ language plpgsql;

create trigger verifica_disponibilita_sede
before insert or update on conferenza
for each row
execute function check_sede_libera();
\end{lstlisting}
\subsection{\texttt{Check\_Sala\_Disponibile}}\label{trigger:saladisponibile}
Una sala non può ospitare più di una sessione alla volta.
\begin{lstlisting}[caption={\texttt{Check\_sala\_sessione\_unica}}, language=SQL, style=mystyle]
create or replace function check_sala_sessione_unica() returns trigger as $$
begin
    if (new.id_sala is null) then
        return new;
    end if;

    if (new.inizio is null) then
        return new;
    end if;

    if (new.fine is null) then
        return new;
    end if;

    if ( select count(*) 
         from sessione 
         where id_sala = new.id_sala 
            and id_sessione <> new.id_sessione
            and (inizio, fine) overlaps (new.inizio, new.fine)
        ) > 0 then
        raise exception 'La sala non e'' disponibile';
    end if;
    return new;
end;
$$ language plpgsql;

create trigger check_sala_sessione_unica
before insert or update on sessione
for each row
execute function check_sala_sessione_unica();
\end{lstlisting}
\subsection{\texttt{Check\_Organizzatore\_Comitato}}\label{trigger:check_organizzatore_comitato}
Ogni volta che si inserisce un nuovo organizzatore all'interno di un comitato bisogna controllare che questo appartenga ad uno degli enti che organizzano la conferenza.
\begin{lstlisting}[language=SQL, style=mystyle, caption={\texttt{Check\_organizzatori\_comitato}}]
create or replace function check_organizzatore_comitato() returns trigger as $$
declare
    ente_id integer;
begin

    select id_ente into ente_id
    from organizzatore o
    where o.id_organizzatore = new.id_organizzatore;
    
    IF ente_id IS NULL THEN
        RAISE EXCEPTION 'L''organizzatore non esiste';
    END IF;
    
    IF ente_id NOT IN (
        SELECT id_ente
        FROM ente_conferenza
        WHERE id_conferenza IN (
            SELECT id_conferenza
            FROM conferenza
            WHERE NEW.id_comitato IN (comitato_s, comitato_l)
        )
    ) THEN
        RAISE EXCEPTION 'L''organizzatore deve appartenere ad un ente che ha organizzato la conferenza';
    END IF;
    return new;
end;
$$ language plpgsql;

create trigger check_organizzatore_comitato
before insert or update on organizzatore_comitato
for each row
execute function check_organizzatore_comitato();
\end{lstlisting}
\subsection{\texttt{Delete\_Sessioni\_Conferenza}}
Nel caso in cui si volesse modificare la data di inizio o di fine di una conferenza vengono automaticamente cancellate le sessioni che si trovano escluse dal nuovo intervallo di date.
\begin{lstlisting}[language=SQL, style=mystyle, caption={\texttt{Delete\_sessioni\_conferenza}}]
create or replace function delete_sessioni_conferenza() returns trigger as $$
declare
    sessioni_cur cursor for 
    select id_sessione 
    from sessione 
    where id_conferenza = old.id_conferenza;
    sessione_id integer;
begin
    open sessioni_cur;
    loop
        fetch sessioni_cur into sessione_id;
        exit when not found;
        if (select inizio from sessione where id_sessione = sessione_id) < new.inizio 
        OR (select fine from sessione where id_sessione = sessione_id) > new.fine then
            delete from sessione where id_sessione = sessione_id;
        end if;
    end loop;
    close sessioni_cur;
    return new;
end;
$$ language plpgsql;

create trigger delete_sessioni_conferenza
before update on conferenza
for each row
execute function delete_sessioni_conferenza();
\end{lstlisting}
\subsection{\texttt{Check\_Capienza}}\label{trigger:check_capienza}
Ogni volta che si aggiunge un nuovo partecipante della sessione bisogna controllare prima che la capienza della sala dove si svolge la sessione non sia stata raggiunta:
\begin{lstlisting}[language=SQL,caption={\texttt{Check\_Capienza}} style=mystyle]
create or replace function check_capienza_sala() returns trigger as $$
declare
    capienza_s integer;
    partecipanti integer;
begin
    select capienza into capienza_s
    from sala
    where id_sala = new.id_sala;
    
    select count(*) into partecipanti
    from partecipazione
    where id_sessione = new.id_sessione;
    
    if (partecipanti >= capienza_s) then
        raise exception 'La capienza della sala e'' stata raggiunta';
    end if;
    return new;
end;
$$ language plpgsql;

create trigger check_capienza_sala
before insert on partecipazione
for each row
execute function check_capienza_sala();
\end{lstlisting}