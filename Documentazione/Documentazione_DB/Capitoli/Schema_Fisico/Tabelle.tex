In questo capitolo presentiamo l'implementazione fisica della base di dati utilizzando Postgres come sistema di gestione di database. Postgres, o PostgreSQL, è un DBMS open-source con ampie funzionalità, stabilità e una comunità di sviluppatori attiva che offre il supporto completo del linguaggio SQL e fornisce strumenti avanzati per l'ottimizzazione delle query, l'indicizzazione dei dati e la gestione delle transazioni. 
\bigskip

Durante il capitolo, mostreremo il processo di traduzione dello schema logico a quello fisico, concentrandoci su elementi chiave come tabelle, la definizione dei trigger e delle varie procedure al fine di creare un database relazionale efficiente, affidabile e soprattutto attivo che soddisfi le esigenze di un applicativo che possa appoggiarsi su di essa quale piattaforma di gestione dei dati.
\section{Definizione delle tabelle}
\subsection{\textsc{Utente}}

\begin{lstlisting}[language=SQL,style=mystyle,caption={Tabella: Utente}]
create table utente(
    id_utente serial primary key,
    username text not null unique,
    nome text not null,
    cognome text not null,
    titolo titolo_st,
    email text not null unique,
    password text not null,
    id_ente integer references ente(id_ente) on delete cascade
);
\end{lstlisting}
\subsection{\textsc{Ente}}
\begin{lstlisting}[language=SQL,style=mystyle,caption={Tabella: Ente}]
create table ente(
id_ente serial primary key,
nome text not null unique,
sigla varchar(7) not null,
unique (nome,sigla)
);
\end{lstlisting}
\subsection{\textsc{Indirizzo}}
\begin{lstlisting}[language=SQL,style=mystyle,caption={Tabella: Indirizzo}]
create table indirizzo(
    id_indirizzo serial primary key,
    via text not null,
    civico varchar(5) not null,
    cap varchar(5) ,
    city text not null,
    provincia varchar(2) not null,
    nazione text
);
\end{lstlisting}
\subsection{\textsc{Sede}}
\begin{lstlisting}[language=SQL,style=mystyle,caption={Tabella:  Sede}]
CREATE TABLE sede(
create table sede(
    id_sede serial primary key,
    nome text ,
    id_indirizzo integer references indirizzo(id_indirizzo) on delete set null
);
\end{lstlisting}
\subsection{\textsc{Sponsor}}
\begin{lstlisting}[language=SQL,style=mystyle,caption={Tabella: Sponsor}]
create table sponsor(
    id_sponsor serial primary key,
    nome text not null
);
\end{lstlisting}
\subsection{\textsc{Comitato}}
Ogni comitato ha una tipologia che varia tra i valori \textit{scientifico} e \textit{locale}. Definiamo quindi il tipo \texttt{comitato\_st} che useremo per specificare la tipologia del comitato:
\begin{lstlisting}[language=SQL,style=mystyle,caption={Tabella: Comitato}]
create type comitato_st as enum ('locale','scientifico');

create table comitato(
    id_comitato serial primary key,
    tipologia comitato_st not null
);
\end{lstlisting}
\subsection{\textsc{Organizzatore, Speaker, Partecipante}}
\begin{lstlisting}[language=SQL,style=mystyle,caption={Tabella: Organizzatore}]
create type titolo_st as enum ('Dottore','Dottoressa','Professore','Professoressa','Assistente','Ricercatore','Ricercatrice','Ingegnere');

create table organizzatore(
    id_organizzatore serial primary key,
    nome text not null,
    cognome text not null,
    titolo titolo_st,
    email text not null unique,
    id_ente integer references ente(id_ente) on delete cascade
);
\end{lstlisting}
\begin{lstlisting}[language=SQL,style=mystyle,caption={Tabella: Partecipante}]
CREATE TABLE partecipante(
id_partecipante SERIAL PRIMARY KEY,
nome TEXT NOT NULL,
cognome TEXT NOT NULL,
titolo titolo_st,
email TEXT NOT NULL UNIQUE, 
id_ente INTEGER REFERENCES ente(id_ente) ON DELETE SET NULL
);
\end{lstlisting}
\begin{lstlisting}[language=SQL,style=mystyle, caption={Tabella: Speaker}]
create table speaker(
    id_speaker serial primary key,
    nome text not null,
    cognome text not null,
    titolo titolo_st,
    email text not null unique,
    id_ente integer references ente(id_ente) on delete cascade NOT NULL
);

\end{lstlisting}
\subsection{\textsc{Sala}}
\begin{lstlisting}[language=SQL,style=mystyle,caption={Tabella: Sala}]
create table sala(
    id_sala serial primary key,
    nome text not null,
    capienza integer not null,
    id_sede integer references sede(id_sede) on delete cascade
);
\end{lstlisting}
\subsection{\textsc{Conferenza}}
\begin{lstlisting}[language=SQL,style=mystyle, caption={Tabella: Conferenza}]
create table conferenza(
    id_conferenza serial primary key,
    titolo text not null,
    descrizione text not null,
    inizio timestamp not null,
    fine timestamp not null,
    id_sede integer references sede(id_sede) on delete set null,
    comitato_s integer references comitato(id_comitato) on delete set null,
    comitato_l integer references comitato(id_comitato) on delete set null,
    id_utente integer references utente(id_utente) on delete cascade,
    check (inizio <= fine),
    unique (titolo,inizio,fine,id_sede)
);
\end{lstlisting}
\subsection{\textsc{Sessione}}
\begin{lstlisting}[language=SQL,style=mystyle, caption={Tabella: Sessione}]
create table sessione(
    id_sessione serial primary key,
    titolo text not null,
    inizio timestamp not null,
    fine timestamp not null,
    id_coordinatore integer references organizzatore(id_organizzatore) on delete set null,
    id_conferenza integer references conferenza(id_conferenza) on delete cascade,
    id_sala integer references sala(id_sala) on delete set null,
    check (inizio <= fine),
    unique (titolo,inizio,fine,id_conferenza,id_sala)
);
\end{lstlisting}
\subsection{\textsc{Partecipazione}}
\begin{lstlisting}[language=SQL,style=mystyle, caption={Tabella: Partecipazione}]
create table partecipazione(
    id_partecipante integer references partecipante(id_partecipante) on delete cascade,
    id_sessione integer references sessione(id_sessione) on delete cascade,
    unique (id_partecipante,id_sessione) 
);
\end{lstlisting}
\subsection{\textsc{Ente\_Conferenza}}
\begin{lstlisting}[language=SQL,style=mystyle, caption={Tabella: Ente\_Conferenza}]
create table ente_conferenza(
    id_ente integer references ente(id_ente) on delete cascade,
    id_conferenza integer references conferenza(id_conferenza) on delete cascade,
    unique (id_ente,id_conferenza)
);
\end{lstlisting}
\subsection{\textsc{Valuta}}
\begin{lstlisting}[language=SQL,style=mystyle,caption={Tabella: Valuta}]
create table valuta(
    iso char(3) primary key,
    nome text not null,
    simbolo text not null
);
\end{lstlisting}
\subsection{\textsc{Sponsor\_Conferenza}}
\begin{lstlisting}[language=SQL,style=mystyle, caption={Tabella: Sponsor\_Conferenza}]
create table sponsor_conferenza(
    id_sponsor integer references sponsor(id_sponsor) on delete cascade not null,
    contributo numeric(1000,2) not null,
    valuta char(3) references valuta(iso) not null,
    id_conferenza integer references conferenza(id_conferenza) on delete cascade not null,
    unique (id_sponsor,id_conferenza) 
);
\end{lstlisting}
\subsection{\textsc{Programma}}
\begin{lstlisting}[language=SQL,style=mystyle,caption={Tabella: Programma}]
create table programma(
    id_programma serial primary key,
    id_sessione integer references sessione(id_sessione) on delete cascade not null,
    id_keynote integer references speaker(id_speaker) on delete set null,
    unique (id_programma, id_sessione)
);
\end{lstlisting}
\subsection{\textsc{Intervento}}
\begin{lstlisting}[language=SQL,style=mystyle, caption={Tabella: Intervento}]
create table intervento(
    id_intervento serial primary key,
    titolo text not null,
    abstract text not null,
    inizio timestamp not null,
    fine timestamp not null,
    id_speaker integer references speaker(id_speaker) on delete cascade,
    id_programma integer references programma(id_programma) on delete cascade not null,
    unique (id_speaker,titolo,id_programma), 
    check (inizio <= fine) 
);
\end{lstlisting}
\subsection{\textsc{Intervallo}}
\begin{lstlisting}[language=SQL,style=mystyle, caption={Tabella: Programma}]
create type intervallo_st as enum ('pranzo','coffee break');
create table intervallo(
    id_intervallo serial primary key,
    tipologia intervallo_st not null,
    inizio timestamp not null,
    fine timestamp not null,
    check (inizio <= fine), 
    id_programma integer references programma(id_programma) on delete cascade not null
);
\end{lstlisting}
\subsection{\textsc{Evento}}
\begin{lstlisting}[language=SQL,style=mystyle, caption={Tabella: Evento}]
create table evento(
    id_evento serial primary key,
    tipologia text not null,
    inizio timestamp not null,
    fine timestamp not null,
    check (inizio <= fine), 
    id_programma integer references programma(id_programma) on delete cascade not null
);
\end{lstlisting}
\subsection{\textsc{Organizzatore\_Comitato}}
\begin{lstlisting}[language=SQL,style=mystyle, caption={Tabella: Organizzatore\_Comitato}]
create table organizzatore_comitato(
    id_organizzatore integer references organizzatore(id_organizzatore) on delete cascade,
    id_comitato integer references comitato(id_comitato) on delete cascade,
    unique (id_organizzatore,id_comitato) 
);
\end{lstlisting}