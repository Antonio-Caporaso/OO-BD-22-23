\section{Dizionario dei vincoli}\label{sez:costraints}
\begin{longtable}{|p{0.32\textwidth}|p{0.2\textwidth}|p{0.3\textwidth}|p{0.1\textwidth}|}
	\hline
	\textbf{Vincolo} & \textbf{Tipo} & \textbf{Descrizione} & \textbf{Vedi} \\
	\hline
	\endfirsthead
	
	% Intestazione normale
	\multicolumn{4}{l}{\footnotesize\itshape Continua dalla pagina precedente} \\
	\hline
	\textbf{Vincolo} & \textbf{Tipo} & \textbf{Descrizione} & \textbf{Vedi}  \\
	\hline
	\endhead
	
	% Piede normale
	\hline
	\multicolumn{3}{r}{\footnotesize\itshape Continua nella prossima pagina} \\
	\endfoot
	
	% Piede finale
	\hline
	\endlastfoot
	\textsc{Check\_Programma} & Interrelazionale & In un programma non devono esserci eventi, intervalli od interventi che si sovrappongono. & \ref{vincolo:check_programma}\\
	\hline
	\textsc{Check\_Data} & Interrelazionale & La data di inizio e di fine di un intervallo, un intervento o un evento devono essere coerenti con quelli della sessione a cui appartengono. & \ref{trigger:check_data} \\ \hline
	\textsc{Check\_Sala\_Sede} & Interrelazionale & La sala in cui si svolge una sessione deve appartenere alla sede in cui si svolge la conferenza della sessione. & \ref{trigger:salainsede} \\ \hline
	\textsc{Check\_Data\_Sessione} & Interrelazionale & La data di inizio e di fine di ogni sessione deve essere compresa tra l'inizio e la fine della propria conferenza. & \ref{trigger:datasessione} \\ \hline
	\textsc{Check\_Coordinatore} & Interrelazionale & Il coordinatore di una sessione deve appartenere al comitato scientifico della conferenza. & \ref{trigger:check_coordinatore} \\ \hline
	\textsc{Check\_Comitati} & Intrarelazionale & Ogni volta che si modifica la tabella \textsc{Conferenza} bisogna controllare che i valori indicati per i comitati siano coerenti con la tipologia di comitato della colonna. & \ref{trigger:check_comitati}\\ \hline
	\textsc{Check\_Sede\_Disponibile} & Interrelazionale & Quando si inserisce una nuova conferenza bisogna controllare che la sede sia disponibile. Una sede è considerata disponibile se ha almeno una sala non occupata nel periodo di tempo indicato per la conferenza. & \ref{trigger:check_sede_disponibile} \\ \hline
	\textsc{Check\_Sala\_Disponibile} & Interrelazionale & Quando si inserisce una nuova sessione bisogna controllare che la sala indicata sia effettivamente disponibile e non occupata nei giorni indicati. & \ref{trigger:saladisponibile}\\ \hline
	\textsc{Check\_Organizzatori} & Interrelazionale & Gli organizzatori appartenenti ai comitati di una conferenza devono appartenere agli enti che organizzano quella conferenza. & \ref{trigger:check_organizzatore_comitato} \\ \hline
	\textsc{Check\_Capienza} & Interrelazionale & Ogni volta che si aggiunge un nuovo partecipante di una sessione bisogna controllare che non sia stata raggiunta la capienza della sala in cui si svolge la sessione. & \ref{trigger:check_capienza} \\ \hline
	\textsc{Set\_Sale\_Null} & Interrelazionale & Ogni volta che viene modificata la sede di una conferenza bisogna mettere a \textsc{NULL} le sale di ogni sessione presente nella conferenza per rispettare il vincolo \textsc{Check\_Sala\_Sede} & \ref{trigger:setsalenull} \\ \hline
	\textsc{Check\_Keynote} & Interrelazionale & Ogni volta che viene specificato il keynote speaker per una sessione bisogna controllare che questi sia uno speaker previsto nel programma. & \ref{trig:check_keynote} \\ 
	 
	
\end{longtable}