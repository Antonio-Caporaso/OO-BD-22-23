\section{PGConf NPL 2023}
\subsection{Descrizione}
PGConf è una conferenza di tre giorni a Napoli ricca di storie di utenti e best practice su come utilizzare PostgreSQL, il database open source più avanzato al mondo. La conferenza si terrà al Centro Congressi Napoli dal 3 agosto 2023 al 5 agosto 2023 ed è organizzata dalla Federico II di Napoli. Tra gli sponsor figurano aziende del mondo informatico quali Amazon, Apple, Cisco ed IBM.
La conferenza avrà in totale cinque sessioni:
\begin{enumerate}
	\item \textbf{Best practice su PostgreSQL:} il primo giorno dalle 10:30 alle 12:30
	\item \textbf{PostgreSQL è il database del futuro:} il primo giorno dalle 14:30 alle 16:30
	\item \textbf{Performance di PostgreSQL:} il secondo giorno dalle 10:30 alle 12:30
	\item \textbf{PostgreSQL e il mondo del lavoro:} il secondo giorno dalle 14:30 alle 16:30
	\item \textbf{PostgreSQL e Kubernetes:} l'ultimo giorno dalle 10:30 fino alla fine della conferenza.
\end{enumerate}

\begin{lstlisting}[language=SQL, style=mystyle,caption={Conferenza PGConf NPL}]
do $$
declare
	conferenza_id integer; --Id conferenza inserita
	comitato_sc_id integer; --Id comitato scientifico inserito
	comitato_lc_id integer; --Id comitato locale inserito
	sessione1_id integer;
	sessione2_id integer;
	sessione3_id integer;
	sessione4_id integer;
	sessione5_id integer;
	programma1_id integer;
	programma2_id integer;
	programma3_id integer;
	programma4_id integer;
	programma5_id integer;
begin

	call add_conferenza(
	'PGConf NPL 2023', --Titolo conferenza
	'2023-08-03 9:00', --Inizio
	'2023-08-05 12:30',--Fine
	5,
	'PGConf e'' una conferenza di tre giorni a Napoli 
	ricca di storie di utenti e best practice su come utilizzare PostgreSQL, 
	il database open source pie'' avanzato al mondo', --Descrizione
	'UNINA', --Ente organizzatore
	1 --Utente che inserisce la conferenza 
	);

	select c.id_conferenza FROM conferenza c where titolo ='PGConf NPL 2023' and inizio='2023-08-03 9:00' and fine='2023-08-05 12:30' and id_sede=5 into conferenza_id;
	if (not found) then
	    rollback;
	end if;
	
	--Aggiunta sponsorizzazioni
	call add_sponsorizzazione(1,5000.50,'USD',conferenza_id); --Apple
	call add_sponsorizzazione(2,3500.50,'USD',conferenza_id); --Amazon
	call add_sponsorizzazione(17,3500.50,'USD',conferenza_id); --Cisco
	call add_sponsorizzazione(30,3500.50,'USD',conferenza_id); --IBM

	select comitato_s from conferenza where id_conferenza=conferenza_id into comitato_sc_id;
	select comitato_l from conferenza where id_conferenza=conferenza_id into comitato_lc_id;

	--Aggiungo membri comitato
	call add_membri_comitato('1,2',comitato_sc_id); -- Comitato scientifico
	call add_membri_comitato('3,4',comitato_lc_id); -- Comitato locale

	-- Aggiunta sessioni
	sessione1_id = add_new_sessione('Best practice su PostgreSQL','2023-08-03 10:30','2023-08-03 12:30',13,conferenza_id,1);
	sessione2_id = add_new_sessione('PostgreSQL e'' il database del futuro','2023-08-03 14:30','2023-08-03 16:30',14,conferenza_id,2);
	sessione3_id = add_new_sessione('Performance di PostgreSQL','2023-08-04 10:30','2023-08-04 12:30',15,conferenza_id,1);
	sessione4_id = add_new_sessione('PostgreSQL e il mondo del lavoro','2023-08-04 14:30','2023-08-04 18:30',13,conferenza_id,2);
	sessione5_id = add_new_sessione('PostgreSQL e Kubernetes','2023-08-05 10:30','2023-08-05 12:30',14,conferenza_id,1);

-- Specifica programma sessione 1

select id_programma from programma where id_sessione = sessione1_id into programma1_id;

	call add_evento('Apertura sessione',programma1_id,'10 minutes');
	call add_intervento('Ottimizzazione delle query in PostgreSQL',
	' Questo intervento illustrera'' le migliori pratiche per ottimizzare le query in PostgreSQL, inclusi suggerimenti per la scrittura di query efficienti e l''utilizzo degli indici corretti.',
	1,
	programma1_id,
	'30 minutes');
	
	call add_intervento(
	    'Gestione delle transazioni in PostgreSQL',
	    'Questo intervento fornira'' consigli e best practice per la gestione delle transazioni in PostgreSQL, comprese le strategie di commit e rollback, il controllo della concorrenza e l''utilizzo di blocchi di transazioni.',
	    2,
	    programma1_id,
	    '30 minutes');
	    
	call add_intervallo('coffee break', programma1_id, '20 minutes');

	call add_intervento(
	'Monitoraggio e tuning delle prestazioni in PostgreSQL',
	'Questo intervento si concentrera'' sul monitoraggio e tuning delle prestazioni in PostgreSQL, presentando strumenti e tecniche per l''identificazione e risoluzione dei problemi di performance.',
	3,
	programma1_id,
	'30 minutes');

-- Specifica programma sessione 2
	select id_programma from programma where id_sessione = sessione2_id into programma2_id;
	
	call add_evento('Apertura sessione',programma2_id,'10 minutes');
	
	call add_intervento
	(
	    'Le nuove funzionalita'' di PostgreSQL15',
	    'Questo intervento presentera'' le nuove funzionalita'' introdotte nella versione 14 di PostgreSQL e come queste possono influenzare il futuro dello sviluppo e dell''utilizzo del database.',
	    4,
	    programma2_id,
	    '30 minutes'
	);
	    
	call add_intervento
	(
	    'PostgreSQL e il mondo del cloud',
	    'Questo intervento esplorera'' l''integrazione di PostgreSQL con le piattaforme di cloud computing e le implicazioni di questa evoluzione per le applicazioni e l''adozione del database nel futuro.',
	    5,
	    programma2_id,
	    '30 minutes'
	    );
	    
	call add_intervallo('coffee break', programma2_id, '20 minutes');
	
	call add_intervento(
	    'PostgreSQL e l''intelligenza artificiale',
	    'Questo intervento illustrera'' come PostgreSQL pua'' essere utilizzato nell''ambito dell''intelligenza artificiale, compresi casi d''uso, funzionalita'' e best practice per l''integrazione di PostgreSQL con strumenti e librerie di AI.',
	    6,
	    programma2_id,
	    '30 minutes'
	);
	    
-- Specifica programma sessione 3

select id_programma from programma where id_sessione = sessione3_id into programma3_id;

call add_evento('Apertura sessione',programma3_id,'10 minutes');

call add_intervento(
    'Strategie di indicizzazione per migliorare le prestazioni in PostgreSQL',
    'Questo intervento esplorera'' diverse strategie di indicizzazione per ottimizzare le prestazioni delle query in PostgreSQL, compreso l''utilizzo di indici multi-colonna, indici parziali e indici funzionali',
    7,
    programma3_id,
    '30 minutes');
    
call add_evento('Sessione di domande e risposte',programma3_id,'30 minutes');

call add_intervallo('coffee break', programma3_id, '20 minutes');

call add_intervento(
    'Tuning del server PostgreSQL per prestazioni ottimali',
    'Questo intervento si concentrera'' sulla configurazione e ottimizzazione del server PostgreSQL per garantire prestazioni ottimali, compreso il tuning dei parametri di configurazione e la gestione della memoria.',
    8,
    programma3_id,
    '30 minutes');
    
call add_intervento(
    'Monitoraggio delle prestazioni in PostgreSQL',
    'Questo intervento illustrera'' gli strumenti e le tecniche per il monitoraggio delle prestazioni in PostgreSQL, compresi i metodi per identificare e risolvere i colli di bottiglia e i problemi di prestazioni.',
    9,
    programma3_id,
    '30 minutes');
    
-- Specifica programma sessione 4
select id_programma from programma where id_sessione = sessione4_id into programma4_id;

call add_evento('Apertura sessione',programma4_id,'10 minutes');

call add_intervento(
    'Vantaggi di PostgreSQL nel contesto aziendale',
    'Questo intervento fornira'' un''analisi dei vantaggi di utilizzare PostgreSQL nel contesto aziendale, compresa l''affidabilita'', le prestazioni e il supporto della comunita''.',
    10,
    programma4_id,
    '30 minutes');
    
call add_intervento(
    'PostgreSQL e il mondo del cloud computing',
    'Questo intervento esplorera'' come PostgreSQL si integra nel mondo del cloud computing e come le aziende possono trarre vantaggio dall''utilizzo di PostgreSQL come database cloud-native.',
    11,
    programma4_id,
    '30 minutes');
    
call add_evento('Sessione di domande e risposte',programma4_id,'30 minutes');

call add_evento('Networking e dibattito',programma4_id,'30 minutes');

call add_intervento(
    'PostgreSQL e le opportunita'' di lavoro',
    'Questo intervento fornira'' informazioni sulle opportunita'' di lavoro e carriera legate all''esperienza con PostgreSQL, inclusi i ruoli professionali, le competenze richieste e le tendenze del mercato.',
    12,
    programma4_id,
    '30 minutes');
    
call add_intervento(
    'Successo delle aziende con PostgreSQL',
    'Questo intervento presentera'' casi di successo di aziende che hanno adottato PostgreSQL come database principale e i benefici che hanno ottenuto.',
    13,
    programma4_id,
    '30 minutes');
    
call add_intervallo('coffee break', programma4_id, '20 minutes');

call add_intervento(
    'PostgreSQL e le tecnologie emergenti',
    'Questo intervento esplorera'' come PostgreSQL si integra con le tecnologie emergenti, come l''intelligenza artificiale, l''Internet of Things (IoT) e l''analisi dei big data.',
    14,
    programma4_id,
    '30 minutes');
    
-- Specifica programma sessione 5

select id_programma from programma where id_sessione = sessione5_id into programma5_id;

call add_evento('Apertura sessione',programma5_id,'10 minutes');

call add_intervento(
    'Introduzione a Kubernetes per gli amministratori di database',
    'Questo intervento fornira'' un''introduzione a Kubernetes e spieghera'' come gli amministratori di database possono utilizzare questa piattaforma per orchestrare e gestire il deployment di istanze di PostgreSQL.',
    15,
    programma5_id,
    '30 minutes');
    
call add_intervento(
    'Implementazione di PostgreSQL su Kubernetes',
    'Questo intervento illustrera'' come implementare PostgreSQL su un cluster Kubernetes, fornendo linee guida e best practice per il deployment, la scalabilita'' e la gestione delle istanze di PostgreSQL.',
    16,
    programma5_id,
    '30 minutes');
    
call add_evento('Sessione di domande e risposte',programma5_id,'20 minutes');

call add_intervallo('pranzo', programma5_id, '30 minutes');

exception
    when others then
        raise notice 'Errore inserimento conferenza: %',sqlerrm;
        rollback;
end $$;
\end{lstlisting}