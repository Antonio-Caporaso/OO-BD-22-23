\chapter{Definizione del problema}
\section{Traccia}
Si sviluppi un sistema informativo, composto da una base di dati relazionale e da un applicativo Java dotato
di GUI (Swing o JavaFX), per la gestione di \textbf{conferenze scientifiche}. 
\bigskip

Ogni \textbf{conferenza} ha una \textit{data di inizio} e di
\textit{fine}, una \textit{collocazione} (\textit{sede, indirizzo}), uno o più \textit{\textbf{enti}} che la organizzano, degli \textbf{\textit{sponsor}} (che coprono in parte le \textit{spese}), una \textit{descrizione}, ed un \textit{gruppo di \textbf{organizzatori}}, che può essere distinto in \textbf{\textit{comitato scientifico}} e \textbf{\textit{comitato locale}} (che si occupa cioè della logistica). Di ognuno degli organizzatori, così come di tutti i \textbf{partecipanti}, si riportano \textit{titolo, nome, cognome, email} ed \textit{istituzione di afferenza}. 
\bigskip

Ogni conferenza può avere una o più \textbf{sessioni}, anche in parallelo fra loro. Ogni sessione ha una \textit{locazione} all'interno della sede. Per ogni
sessione c'è un \textbf{\textit{programma}}, che prevede la presenza di un \textit{coordinatore (chair)} che gestisce la sessione, ed eventualmente di un \textit{keynote speaker} (un partecipante di particolare rilievo invitato dagli organizzatori). Ogni sessione avrà quindi una successione di \textbf{interventi} ad \textit{orari predefiniti} e di \textit{specifici partecipanti}. Per ogni intervento si conserva un \textit{abstract} (un breve testo in cui viene spiegato il contenuto del lavoro presentato).
\bigskip

Si deve poter considerare la presenza di spazi di intervallo (coffee breaks, pranzo) ma anche la presenza di eventi sociali (cene, gite, etc).
\section{Output attesi dal committente}
\begin{enumerate}
\item Documento di Design della base di dati:
\begin{enumerate}
\item Class Diagram della base di dati.
\item Dizionario delle Classi, delle Associazioni e dei Vincoli.
\item Schema Logico con descrizione di Trigger e Procedure individuate.
\end{enumerate}
\item File SQL contenenti:
\begin{enumerate}
\item Creazione della struttura della base di dati.
\item Popolamento del DB.
\item (Facoltativo, ma apprezzato) README contenente i commenti all’SQL.
\end{enumerate}
\end{enumerate}