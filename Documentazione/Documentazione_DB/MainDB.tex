\documentclass[a4paper,oneside, 11pt]{book}
\usepackage{graphicx}
\usepackage{float}
\usepackage{adjustbox}
\usepackage{listings}
\usepackage{setspace}
\usepackage[labelfont={bf},textfont=it]{caption}
\usepackage{fullpage}
\usepackage{booktabs}
\usepackage{subfig}
\usepackage{enumerate}
\usepackage{pgfplots}
\pgfplotsset{compat=1.8}
\usepackage{tabularx}
\usepackage{multirow}
\usepackage[T1]{fontenc}
\usepackage{inputenc}
\usepackage{hyperref}
\usepackage[italian]{babel}
\usepackage{longtable}
\usepackage{import} 
\usepackage{mathtools}
\usepackage{amssymb,amsmath,amsthm}
\setlength{\parindent}{0pt}
\definecolor{codegreen}{rgb}{0,0.6,0}
\definecolor{codegray}{rgb}{0.5,0.5,0.5}
\definecolor{codepurple}{rgb}{0.58,0,0.82}
\definecolor{backcolour}{rgb}{0.95,0.95,0.92}
\author{\begin{tabular}{cc}Caporaso Antonio & Di Fusco Giorgio \\
\texttt{N86003458} & \texttt{N86004389} \\ \end{tabular}}
\lstdefinestyle{mystyle}{
    backgroundcolor=\color{backcolour},   
    commentstyle=\color{codegreen},
    keywordstyle=\color{magenta},
    numberstyle=\tiny\color{codegray},
    stringstyle=\color{codepurple},
    basicstyle=\ttfamily\footnotesize,
    breakatwhitespace=false,         
    breaklines=true,                 
    captionpos=b,                    
    keepspaces=true,                 
    numbers=left,                    
    numbersep=5pt,                  
    showspaces=false, 
    showlines=false,               
    showstringspaces=false,
    showtabs=false,                  
    tabsize=2,
    morekeywords={*,call, REFERENCES, enable, disable, ENABLED, DISABLED, IS, BEFORE, AFTER, INCREMENT, START, raice, notice, procedure, others, WITH, setof, QUERY, SEQUENCE, FOR, EACH, ROW, INSTEAD OF,serial,TYPE, ENUM, FOR EACH STATEMENT, FUNCTION, LANGUAGE, RETURNS, sqlerrm, RETURN, DECLARE, PLPGSQL, BEGIN, REPLACE, text, OPEN, LOOP, IF, CLOSE, FETCH, EXIT, RAISE}
}
\lstset{
style= mystyle
}
\makeatother
\addto\captionsitalian{%
	\renewcommand{\lstlistingname}{Listato}
	\renewcommand{\lstlistlistingname}{Elenco dei listati}
}
\def\ojoin{\setbox0=\hbox{$\bowtie$}%
  \rule[-.02ex]{.25em}{.4pt}\llap{\rule[\ht0]{.25em}{.4pt}}}
\def\leftouterjoin{\mathbin{\ojoin\mkern-5.8mu\bowtie}}
\def\rightouterjoin{\mathbin{\bowtie\mkern-5.8mu\ojoin}}
\def\fullouterjoin{\mathbin{\ojoin\mkern-5.8mu\bowtie\mkern-5.8mu\ojoin}}
{\renewcommand{\arraystretch}{1.3}%

\begin{document}
\pagestyle{empty}
\begin{titlepage}
	\begin{center}
		\setstretch{1.2}
		\setlength{\parskip}{2ex}
		
		
		\Large\textsc{Università degli Studi di Napoli Federico II}
		
		\includegraphics[width=3cm]{Immagini/logo-federico-II.pdf}
		
		\Large\textsc{Scuola Politecnica e delle Scienze di Base}
		
		\large\textsc{Dipartimento di Ingegneria Elettrica e Tecnologie dell'Informazione}
		
		\large\textsc{Corso di Laurea Magistrale in Informatica}
		
		\textsc{Progetto d'esame di Basi di Dati}
		\vfill
		\setstretch{2}
		\huge\textsc{Progettazione ed implementazione di una base di dati relazionale per la gestione di conferenze scientifiche}
		\vfill
		\setstretch{1}
		\begin{minipage}[t]{.49\textwidth}
			\large
			
			\textbf{Relatrice}\par
			Professoressa Mara \textsc{Sangiovanni}
		\end{minipage}\hfill
	\begin{minipage}[t]{.45\textwidth}
		\large
		\hspace{3.3cm}\textbf{Candidati}\par
		\hfill\begin{tabular}{l}
			 Antonio \textsc{Caporaso} \\ 
			 matr: \texttt{N86003458} \\
			 Giorgio \textsc{Di Fusco} \\
			  matr: \texttt{N86004389} \\
		 \end{tabular}
	\end{minipage}
\vfill
		
		\large Anno Accademico 2022-2023
	\end{center}
\end{titlepage}
\tableofcontents
\listoffigures
\listoftables
\lstlistoflistings
\chapter{Traccia}
\import{Capitoli/Traccia}{Traccia.tex}
\chapter{Progettazione}
\import{Capitoli/Progettazione_Concettuale}{Progettazione_Concettuale.tex}
\import{Capitoli/Schema_Ristrutturato}{Schema_ristrutturato.tex}
\import{Capitoli/Schema_Logico}{Schema_logico.tex}
\chapter{Implementazione fisica}
\import{Capitoli/Schema_Fisico}{Tabelle.tex}
\import{Capitoli/Schema_Fisico}{Trigger.tex}
\import{Capitoli/Schema_Fisico}{Funzioni.tex}
\import{Capitoli/Schema_Fisico}{Viste.tex}
\chapter{Un esempio d'uso}
\import{Capitoli/Use_Case}{PGConf.tex}
\appendix
\chapter{Dizionari}
\import{Capitoli/Dizionari}{Data_dictionary}
\import{Capitoli/Dizionari}{Relationship_dictionary}
\import{Capitoli/Dizionari}{Constraints_dictionary}
\end{document}
